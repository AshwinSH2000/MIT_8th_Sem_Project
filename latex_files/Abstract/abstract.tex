%================Thesis Abstract===============
\begin{center}
\Large{\textbf{ABSTRACT}}\\
\end{center}
%=============================================

The constantly increasing vehicles on the road and the carelessness of the drivers, result in huge number of accidents and fatalities all over the globe. In addition to this, people spend a lot of their time on roads driving their vehicles in busy traffic. Because of these, autonomous vehicles have garnered momentous attention. 
This will also address several enduring challenges related to traffic congestion, vehicle parking, and many other issues in the transportation sector. 
From the various recent literature, it is found that to tackle these issues, Reinforcement Learning, one of the finest Machine learning applications is the best-suited approach. 

To develop an Autonomous Driving Strategy, Deep Deterministic Policy Gradient, one of the RL algorithms is used. This algorithm deals with continuous state spaces, which is the essential requirement of autonomous driving.  Out of the various simulators available, the TORCS simulator satisfies the requirements to implement this project. 

The objectives of lane keep  operation of the car using Reinforcement Learning and avoiding collisions by detecting obstacles using sensors have been achieved successfully. After 14000 episodes, the simulated car learns to drive by itself without any hiccups. Elderly people, patients on medication, persons with disabilities, and busy executives can make use of this technology.


%The abstract is brief synopsis of the project work and should be written in 3  paragraphs. The first paragraph should introduce the area of the topic and give importance of the work / topic in the present day scenario, hence leading to the objective of the project work. The second paragraph may  discuss briefly about the design/ methodology that was adopted in addition to the tools used. The third paragraph should discuss briefly the important results that were obtained and its significance. You may also  discuss about  the important conclusion(s) of the project work. (The abstract should fit in one page only)

% Refer to ACM Computing Classification System Taxonomy, read first page and following the instruction given therein.=============
\par 
\textbf{[Computing Methodologies]}: Machine Learning—learning paradigms, reinforcement learning

