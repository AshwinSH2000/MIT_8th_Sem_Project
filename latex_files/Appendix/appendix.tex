\begin{appendix}

\chapter{Code (if required)}
\section{Kerberos Protocol}
my name is ashwin s h
\scriptsize{
\begin{verbatim}
MODULE main

#my name is ashwin s h

VAR

--Creating agents which are to type agtype
agA : agtype; 
agB : agtype;
agS : agtype;
agI : agtype;
Iactive: boolean;

--Assigning initial to values to all variables

ASSIGN
init(agA.state):=wait;
init(agB.state):=wait;
init(agS.state):=wait;
init(agI.state):=wait;
init(agA.count):=0;
init(agB.count):=0;
init(agS.count):=0;
init(agI.count):=0;
init(agA.authenticated):=FALSE;
init(agB.authenticated):=FALSE;
init(agI.authenticated):=FALSE;
init(agS.authenticated):=TRUE;

--Transitions for the variable indicating presence or absence of intruder

next(Iactive):=
case
!Iactive:{0,1};
Iactive & agI.state=receive4beta:{0,1};
1:Iactive;
esac;

--Transitions for agent A's state

next(agA.state):=
case
agA.state=wait: send1;
agS.state=send2 & agA.state=send1: receive2;
agA.state=receive2: send3alpha;
agB.state=send4alpha & agA.state=send3alpha: receive4alpha;
agA.state=receive4alpha: wait;
1:agA.state;
esac;

--Transitions for agent B's state

next(agB.state):=
case
agA.state=send3alpha & agB.state=wait: receive3alpha;
agI.state=send3beta & agB.state=wait & Iactive: receive3beta;
agI.state=send3beta & agB.state=send4alpha & Iactive: receive3beta;
agB.state=receive3alpha:send4alpha;
agB.state=receive3beta:send4beta;
agB.state=send4alpha:wait;
1:agB.state;
esac;

--Transitions for Server S's state

next(agS.state):=
case
agA.state=send1 & agS.state=wait: receive1;
agS.state=receive1:send2;
agS.state=send2:wait;
1:agS.state;
esac;

--Transitions for the Intruder's state

next(agI.state):=
case
agI.state=wait & agA.state=send3alpha & agB.state=wait & Iactive: receive3beta;
agI.state=receive3beta & Iactive: send3beta;
agI.state=send3beta & agB.state=send4beta & Iactive  : receive4beta;
agI.state=receive4beta & Iactive: wait;
1:agI.state;
esac;

--Transitions for Agent A's counter

next(agA.count):=
case
agA.state=send1|agA.state=receive2: agA.count;
agA.state=send3alpha & agA.count<1:agA.count+1;
agA.count=1 & agA.state=receive2: 0;
1:agA.count;
esac;

--Transitions for Agent B's counter

next(agB.count):=
case
agB.state=receive3beta & agB.count<2|agB.state=receive3alpha & agB.count<2: agB.count+1;
agB.state=send4alpha |agB.state=send4beta:agB.count;
agB.count=1 & agA.state=receive4alpha & !Iactive:0;
agB.count=2 & agA.state=send3alpha|agB.count=1 & agA.state=send3alpha: 0;
1:agB.count;
esac;

--Transitions for Agent I's counter

next(agI.count):=
case
agI.state=receive3beta & agI.count<2 & Iactive:agI.count+1;
agI.state=send3beta & Iactive:agI.count;
agI.state=receive4beta & agI.count<2 & Iactive: agI.count+1;
agI.count=2: 0;
1:agI.count;
esac;

--Transitions for variable indicating agent A's authentication

next(agA.authenticated):=
case
agA.state=receive4alpha :TRUE;
1:agA.authenticated;
esac;

--Transitions for variable indicating agent B's authentication

next(agB.authenticated):=
case
agB.state=send4alpha |agB.state=send4beta :TRUE;
1:agB.authenticated;
esac;

--Transitions for variable indicating agent B's authentication which 
--indicates that it has received the fourth message

next(agI.authenticated):=
case
agI.state=receive4beta :TRUE;
1:agI.authenticated;
esac;

--Agent S always is authenticated so transitions to the false state do not occur

next(agS.authenticated):=
case
1:agS.authenticated;
esac;

--Specifications which detect the presence of replay attack

--Agent B should not receive more messages than what agent A has sent it
--SPEC AG!(agA.count < agB.count);

--Agent I should never receive the fourth message
SPEC AG!(agI.state=receive4beta);

--Module for each agent's type which includes the agent's state variable, 
--its counter and its authentication variable

MODULE agtype

VAR
state: {wait, send1, receive1, send2,receive2, 
send3alpha, send3beta, receive3alpha, receive3beta, 
send4alpha,send4beta, receive4alpha, receive4beta };
count:{0,1,2};
authenticated:boolean;

\end{verbatim}


\section{Kerberos Protocol with Freshness Concept}
\begin{verbatim}
MODULE main

VAR

--Creating agents which are to type agtype
agA : agtype; 
agB : agtype;
agS : agtype;
agI : agtype;
Iactive: boolean;
Fresh:0..20;
Time:0..20;

--Assigning initial to values to all variables
 
ASSIGN
init(agA.state):=wait;
init(agB.state):=wait;
init(agS.state):=wait;
init(agI.state):=wait;
init(agA.count):=0;
init(agB.count):=0;
init(agS.count):=0;
init(agI.count):=0;
init(agA.authenticated):=FALSE;
init(agB.authenticated):=FALSE;
init(agI.authenticated):=FALSE;
init(agS.authenticated):=TRUE;
init(Fresh):=0;
init(Time):=0;

--Transitions for the variable indicating presence or absence of intruder

next(Iactive):=
case
!Iactive:{0,1};
Iactive & agI.state=receive4beta:{0,1};
1:Iactive;
esac;

--Transitions for agent A's state

next(agA.state):=
case
agA.state=wait: send1;
agS.state=send2 & agA.state=send1: receive2;
agA.state=receive2: send3alpha;
agB.state=send4alpha & agA.state=send3alpha: receive4alpha;
agA.state=receive4alpha: wait;
1:agA.state;
esac;

--Transitions for agent B's state

next(agB.state):=
case
agA.state=send3alpha & agB.state=wait & Fresh=0: receive3alpha;
agI.state=send3beta & agB.state=wait & Iactive & Fresh=0: receive3beta;
agI.state=send3beta & agB.state=send4alpha & Iactive & Fresh=0: receive3beta;
agB.state=receive3alpha:send4alpha;
agB.state=receive3beta:send4beta;
agB.state=send4alpha:wait;
1:agB.state;
esac;

--Transitions for Server S's state

next(agS.state):=
case
agA.state=send1 & agS.state=wait: receive1;
agS.state=receive1:send2;
agS.state=send2:wait;
1:agS.state;
esac;

--Transitions for the Intruder's state

next(agI.state):=
case
agI.state=wait & agA.state=send3alpha & agB.state=wait & Iactive: receive3beta;
agI.state=receive3beta & Iactive: send3beta;
agI.state=send3beta & agB.state=send4beta & Iactive  : receive4beta;
agI.state=receive4beta & Iactive: wait;
agI.state=send3beta & Time>2: wait;
1:agI.state;
esac;

--Transitions for Agent A's counter

next(agA.count):=
case
agA.state=send1|agA.state=receive2: agA.count;
agA.state=send3alpha & agA.count<1:agA.count+1;
agA.count=1 & agA.state=receive2: 0;
1:agA.count;
esac;

--Transitions for Agent B's counter

next(agB.count):=
case
agB.state=receive3beta & agB.count<2|agB.state=receive3alpha & agB.count<2: agB.count+1;
agB.state=send4alpha|agB.state=send4beta:agB.count;
agB.count=1 & agA.state=receive4alpha & !Iactive:0;
agB.count=2 & agA.state=send3alpha|agB.count=1 & agA.state=send3alpha: 0;
1:agB.count;
esac;

--Transitions for Agent I's counter

next(agI.count):=
case
agI.state=receive3beta & agI.count<2 & Iactive:agI.count+1;
agI.state=send3beta & Iactive:agI.count;
agI.state=receive4beta & agI.count<2 & Iactive: agI.count+1;
agI.count=2: 0;
1:agI.count;
esac;

--Transitions for variable indicating agent A's authentication

next(agA.authenticated):=
case
agA.state=receive4alpha :TRUE;
1:agA.authenticated;
esac;

--Transitions for variable indicating agent B's authentication

next(agB.authenticated):=
case
agB.state=send4alpha|agB.state=send4beta :TRUE;
1:agB.authenticated;
esac;

--Transitions for variable indicating agent B's authentication which 
--indicates that it has received the fourth message

next(agI.authenticated):=
case
agI.state=receive4beta :TRUE;
1:agI.authenticated;
esac;

--Agent S always is authenticated so transitions to the false state do not occur

next(agS.authenticated):=
case
1:agS.authenticated;
esac;

--Transitions for the freshness variable

next(Fresh):=
case
agA.state=send3alpha & agB.state=wait:0;
Fresh<20:Fresh+1;
1:0;
esac;

--Transitions for the Intruder's timer variable

next(Time):=
case
agI.state=receive3beta:0;
Time<20:Time+1;
1:0;
esac;

--Specifications which detect the presence of replay attack

--Agent B should not receive more messages than what agent A has sent it
SPEC AG!(agA.count < agB.count);

--Agent I should never receive the fourth message
SPEC AG!(agI.state=receive4beta);

--Module for each agent's type which includes the agent's state variable, 
--its counter and its authentication variable

MODULE agtype

VAR
state:{wait, send1, receive1, send2,receive2, 
send3alpha, send3beta, receive3alpha, receive3beta, 
send4alpha, send4beta, receive4alpha, receive4beta};
count:{0,1,2};
authenticated:boolean;

\end{verbatim}

\chapter{Trace Files}

\section{Replay Attack}
\begin{verbatim}
-- specification AG !(agA.count < agB.count)  is false
-- as demonstrated by the following execution sequence
Trace Description: CTL Counterexample 
Trace Type: Counterexample 
-> State: 1.1 <-
  agA.state = wait
  agA.count = 0
  agA.authenticated = 0
  agB.state = wait
  agB.count = 0
  agB.authenticated = 0
  agS.state = wait
  agS.count = 0
  agS.authenticated = 1
  agI.state = wait
  agI.count = 0
  agI.authenticated = 0
  Iactive = 0
-> Input: 1.2 <-
-> State: 1.2 <-
  agA.state = send1
  agS.count = 2
-> Input: 1.3 <-
-> State: 1.3 <-
  agS.state = receive1
-> Input: 1.4 <-
-> State: 1.4 <-
  agS.state = send2
-> Input: 1.5 <-
-> State: 1.5 <-
  agA.state = receive2
  agS.state = wait
-> Input: 1.6 <-
-> State: 1.6 <-
  agA.state = send3alpha
  Iactive = 1
-> Input: 1.7 <-
-> State: 1.7 <-
  agA.count = 1
  agB.state = receive3alpha
  agI.state = receive3beta
-> Input: 1.8 <-
-> State: 1.8 <-
  agB.state = send4alpha
  agB.count = 1
  agI.state = send3beta
  agI.count = 1
-> Input: 1.9 <-
-> State: 1.9 <-
  agA.state = receive4alpha
  agB.state = receive3beta
  agB.authenticated = 1
-> Input: 1.10 <-
-> State: 1.10 <-
  agA.state = wait
  agA.authenticated = 1
  agB.state = send4beta
  agB.count = 2
-- specification AG !(agI.state = receive4beta)  is false
-- as demonstrated by the following execution sequence
Trace Description: CTL Counterexample 
Trace Type: Counterexample 
-> State: 2.1 <-
  agA.state = wait
  agA.count = 0
  agA.authenticated = 0
  agB.state = wait
  agB.count = 0
  agB.authenticated = 0
  agS.state = wait
  agS.count = 0
  agS.authenticated = 1
  agI.state = wait
  agI.count = 0
  agI.authenticated = 0
  Iactive = 0
-> Input: 2.2 <-
-> State: 2.2 <-
  agA.state = send1
  agS.count = 2
-> Input: 2.3 <-
-> State: 2.3 <-
  agS.state = receive1
-> Input: 2.4 <-
-> State: 2.4 <-
  agS.state = send2
-> Input: 2.5 <-
-> State: 2.5 <-
  agA.state = receive2
  agS.state = wait
-> Input: 2.6 <-
-> State: 2.6 <-
  agA.state = send3alpha
  Iactive = 1
-> Input: 2.7 <-
-> State: 2.7 <-
  agA.count = 1
  agB.state = receive3alpha
  agI.state = receive3beta
-> Input: 2.8 <-
-> State: 2.8 <-
  agB.state = send4alpha
  agB.count = 1
  agI.state = send3beta
  agI.count = 1
-> Input: 2.9 <-
-> State: 2.9 <-
  agA.state = receive4alpha
  agB.state = receive3beta
  agB.authenticated = 1
-> Input: 2.10 <-
-> State: 2.10 <-
  agA.state = wait
  agA.authenticated = 1
  agB.state = send4beta
  agB.count = 2
-> Input: 2.11 <-
-> State: 2.11 <-
  agA.state = send1
  agI.state = receive4beta

\end{verbatim}

\section{Replay Attack overcome using Freshness Concept}

\begin{verbatim}
-- specification AG !(agA.count < agB.count)  is true
-- specification AG !(agI.state = receive4beta)  is true

\end{verbatim}

}
\end{appendix}
