\chapter{Conclusion and Future Scope}

\section{Conclusion}

Today, self driving cars is the need of the hour because of the increasing traffic on the roads. Having self driven cars with highest possible efficiency and reliability are required very much for safe travelling. Even though many researchers work on this topic by making use of AI/ML tools, still 100\% efficiency has not been achieved. 
In this work, one of the Reinforcement Learning algorithms, DDPG is used for controlled driving of a car. In the second phase, object detection and collision aviodance is achieved with the help of various sensors present on the car. Thus, the agent can handle situations like front, side obstacle detection, overtaking and can  maneuver itself to avoid collision. 

Vehicles have become an integral part of our life. Just a few decades ago, autonomous driving was just a distant dream. Over the past few years, there has been such a tremendous improvement in technology which is enabling us to bring such dreams into reality. 

Elderly people, patients on medication, persons with disabilities who will have to depend on other persons for mobility can now move around independently.
In addition, busy executives can attend to their work undisturbed while travelling and this saves their time. With more and more autonomous vehicles on the road, it can be hoped that traffic accidents can be reduced to a minimum since all these vehicles can be designed to strictly follow  the traffic rules. 


\section{Future Scope}
This project does not take into account the changes in weather conditions, illumination levels etc. There can be some variations in the sensor values if one of more of these change and it can hamper the performance. It needs to be checked if all these sensors give accurate values after changes in weather and light. If not, it needs to be calibrated accordingly. 

Since the simulator has no provision to include traffic signals, the same could not be implemented. Hence training the agent to sense the signals and act accordingly is a major task for the future. 

The agent heavily relies on lane markings to detect lanes. In the real world, if the lane markings are erased off in certain areas or if the lane markings are not at all there, it struggles to maintain its lane and drive efficiently. In the future, it can be trained to use a camera to detect the edge of the road and perform seamless driving. 



%This chapter should include 
%\begin{itemize}
%\item Brief summary of the work -  Problem statement / objective, in brief, Work methodology adopted, in brief
%\item Conclusions - 	General conclusions, Significance of the results obtained
%\item Future scope of work.
%\end{itemize}

